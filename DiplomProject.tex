\documentclass[a4paper,12pt]{report}
\usepackage[utf8]{inputenc}
\usepackage{polyglossia}
\setdefaultlanguage{russian}
\usepackage{amssymb,amsfonts,amsmath,mathtext,cite,enumerate,float}



\usepackage[toc]{appendix}
\usepackage{titletoc}
\usepackage{tocloft}
\usepackage{etoolbox}
\usepackage{textcase}
\usepackage{titlesec}
\usepackage{fancyhdr}

\usepackage{fontspec}
\setmainfont{Times New Roman}



\usepackage{geometry} % Меняем поля страницы
\geometry{left=2cm}% левое поле
\geometry{right=1.5cm}% правое поле
\geometry{top=2cm}% верхнее поле
\geometry{bottom=2cm}% нижнее поле






%Титульный лист включается в общую нумерацию, но на самом титульном листе номер не ставится. На последующих страницах номер проставляется в правом верхнем углу. 

\fancyhf{}
\fancyhead[R]{\thepage}
\pagestyle{fancy}

% redefine the plain pagestyle
\fancypagestyle{plain}{%
\fancyhf{} % clear all header and footer fields
\fancyhead[R]{\thepage} % except the center
}
\renewcommand{\headrulewidth}{0pt}

\makeatletter


%chapter - Делаем текст большими буквами и с точками в содержании
\renewcommand*\l@chapter[2]{%
  \ifnum \c@tocdepth >\m@ne
    \addpenalty{-\@highpenalty}%
    \vskip 1.0em \@plus\p@
    \setlength\@tempdima{1.5em}%
    \begingroup
      \parindent \z@ \rightskip \@pnumwidth
      \parfillskip -\@pnumwidth
      \leavevmode
      \advance\leftskip\@tempdima
      \hskip -\leftskip
      \MakeTextUppercase{#1}\nobreak\leaders\hbox{$\m@th
\mkern \@dotsep mu\hbox{.}\mkern \@dotsep
mu$}\hfill
\nobreak\hb@xt@\@pnumwidth{\hss #2}\par
      \penalty\@highpenalty
    \endgroup
  \fi}




%http://tex.stackexchange.com/questions/63390/how-to-decrease-spacing-before-chapter-title
%since \@makechapterhead adds a 50pt space above the title and 40pt after it.
%A different strategy might be to redefine \@makechapterhead yourself:

\def\@makechapterhead#1{%
  %%%%%\vspace*{50\p@}% %%% removed!
  {\parindent \z@ \raggedright
    \normalfont
    \interlinepenalty\@M
    \large \bfseries \thechapter.  \MakeUppercase{#1}\par\nobreak
    \vskip 10\p@
  }}

%Со звездочкой
\def\@makeschapterhead#1{%
  %%%%%\vspace*{50\p@}% %%% removed!
  {\parindent \z@ \raggedright
    \normalfont
    \interlinepenalty\@M
    \Large \bfseries \MakeUppercase{#1}\par\nobreak
    \vskip 10\p@
  }}



\titleformat*{\section}{\large\bfseries\itshape}
\renewcommand\numberline[1]{#1.\hskip0.7em}


\makeatother




\renewcommand{\theenumi}{\arabic{enumi}}% Меняем везде перечисления на цифра.цифра
\renewcommand{\labelenumi}{\arabic{enumi}.}% Меняем везде перечисления на цифра.цифра
\renewcommand{\theenumii}{.\arabic{enumii}}% Меняем везде перечисления на цифра.цифра
\renewcommand{\labelenumii}{\arabic{enumi}.\arabic{enumii}.}% Меняем везде перечисления на цифра.цифра
\renewcommand{\theenumiii}{.\arabic{enumiii}}% Меняем везде перечисления на цифра.цифра
\renewcommand{\labelenumiii}{\arabic{enumi}.\arabic{enumii}.\arabic{enumiii}.}% Меняем везде перечисления на цифра.цифра


\usepackage{setspace}
\setstretch{1.5}

\setcounter{tocdepth}{1} %n=1 это chapter и section в оглавлении

\parindent=1cm %абзацный отступ



%Здесь переменные. Они вынесутся в отдельный файл
\newcommand{\username}{Иванов Иван Иванович}
\newcommand{\thesisTheme}{Разработка системы управления миром с использованием систем контроля версий}
\newcommand{\projectManager}{Другой Иванов Иван Иванович, аспирант}
\newcommand{\workplace}{НИЯУ МИФИ} %переменные: тема работы, студент и т.д.


\begin{document}


\begin{titlepage}
\newpage

\begin{center}
\bfseries НАЦИОНАЛЬНЫЙ ИССЛЕДОВАТЕЛЬСКИЙ ЯДЕРНЫЙ УНИВЕРСИТЕТ \guillemotleft МИФИ\guillemotright \\
\vspace{0.7cm}
\bfseries ФАКУЛЬТЕТ КИБЕРНЕТИКИ И ИНФОРМАЦИОННОЙ БЕЗОПАСНОСТИ \\*
КАФЕДРА \guillemotleft КОМПЬЮТЕРНЫЕ СИСТЕМЫ И ТЕХНОЛОГИИ\guillemotright \\*
\end{center}
 
Специальность 230101 \hfill Группа В7-123

\vspace{2em}

\hfill\begin{minipage}[t]{6cm}
\begin{center} 
	\guillemotleft \uppercase{Утверждаю}\guillemotright \\
	Заведующий кафедрой
\end{center}
\begin{flushleft}
\rule{3cm}{0.4pt} М.А. Иванов
"\rule{.5cm}{0.4pt}" \rule{3cm}{0.4pt} 2013 г.
\end{flushleft}
\end{minipage}


\vspace{8em}

\begin{center}
\large\bfseries ЗАДАНИЕ НА ВЫПУСКНУЮ КВАЛИФИКАЦИОННУЮ РАБОТУ \\ (ДИПЛОМНЫЙ ПРОЕКТ)
\end{center}

\vspace{3em}
 
\begin{flushleft}
Фамилия, имя, отчество студента: \textbf{\username} \\*
Тема работы: \begin{minipage}[t]{0.75\textwidth} {\textbf{\thesisTheme}} \end{minipage}
\end{flushleft}
Руководитель работы: \textbf{\projectManager} \\*
Место выполнения: \textbf{\workplace}

\vspace{\fill}

\begin{center}
Москва 2014
\end{center}

\end{titlepage}% это титульный лист
\setcounter{page}{2} %Аннотация - это уже вторая страница

\chapter*{\centerline{Аннотация}}

Пояснительная записка состоит из пяти глав.

В данной работе описываются актуальность темы и постановка задачи с описанием её планируемого функционала, обоснование выбора архитектуры планируемой информационной системы и обзор технологий, которые используются для построения аналогичных систем, описание проектирования системы, описание интерфейсов с примерами скриншотов.

Заключительная глава посвящена тестированию программ, тестированию методов фильтрации данных и проверке корректности работы алгоритмов на физической модели.

В приложении приведен код программы, написанный на bash, языке Java, языке С и makefile.

\endinput% аннотация
\newpage

\chapter*{Техническое задание}
	\addcontentsline{toc}{chapter}{Техническое задание}
	\begin{enumerate}
	\item Исходные данные:
	\mbox{}\\Разрабатываемая система предназначена для стабилизации восьмимоторного квадрокоптера на базе микроконтроллера ATMEGA328P-PU с использованием библиотек Arduino. Моторы квадрокоптера предусматривают вращение только в одну сторону. Система должна стабилизировать полет квадрокоптера.
	\item Содержание задания:
	\begin{enumerate}
		\item\itshape литература и обзор работ, связанных с темой работы
		\item\itshape расчетно-конструкторская, теоретическая, технологическая части
		\item\itshape экспериментальная часть
	\end{enumerate}
	\item Основная литература
	\item Отчетный материал:
		\begin{enumerate}
		\item{\itshape пояснительная записка}
		\item{\itshape макетно-экспериментальная часть:}
			\begin{enumerate}
			\item Листинги отлаженных программ
			\item Материалы отладки
			\item Дистрибутив системы на CD
			\item Инструкция пользователя
			\end{enumerate}
		\end{enumerate}
\end{enumerate}
\vfill %отступаем до конца страницы
\begin{center}
	\begin{minipage}[b]{12cm}
		\begin{flushleft}
		Дата выдачи задания: 15 октября 2013 г. \\
		Задание принял к исполнению \rule{3cm}{0.4pt} \\
		Руководитель \rule{6cm}{0.4pt} \\
		Консультант \rule{6cm}{0.4pt} \\
		\end{flushleft}
	\end{minipage}
\end{center}% текст технического задания


\renewcommand{\cfttoctitlefont}{\Large\bfseries\MakeUppercase}
\setlength{\cftbeforetoctitleskip}{-3em}
\renewcommand{\contentsname}{Содержание}
\tableofcontents % это оглавление, которое генерируется автоматически
\newpage



\chapter*{Введение}
	\addcontentsline{toc}{chapter}{Введение}
	Главной целью введения (3-5 страниц текста) служит определение места представленной в
дипломном проекте темы в ряду аналогичных научно-технических и инженерных разработок. Во
введении излагается формулировка главных научных и инженерных вопросов дипломного проекта,
границы разрабатываемой темы, особенности подхода к решению (выбора метода исследования,
расчета или инженерного решения). Введение завершается развернутой формулировкой основной
цели дипломного проектирования.


\chapter{Обзорная часть}
	\section{Обзор существующих моделей}
	Результаты исследований по вопросам, сформулированным в соответствующих пунктах
	задания на дипломное проектирование. В этом разделе на основе анализа литературных и других
	источников рассматриваются возможные варианты решения поставленной задачи. Дается их
	критическая оценка, обосновываются метод решения, который используется при выполнении
	инженерной разработки темы, и выбранные для этой цели средства.
	
	
	Обзорная часть проекта немыслима без указания ссылок на источники (книги, статьи, фирменные
	документы, материалы из сети Internet). Количество ссылок может характеризовать объем и глубину
	исследования, но объем самой обзорной части никоим образом об этом не свидетельствует. Обзорная
	часть ни в коем случае не должна доминировать в пояснительной записке. Не следует переписывать
	фрагменты из источников, а тем более статьи целиком. Нескольких фраз, написанных самим автором
	обзора и характеризующих тот или иной метод, способ и пр., вполне достаточно.


\chapter{Реализация}
	\section{Разработка структуры программного обеспечения}
Однако пояснительная записка должна быть описанием того, как делалась работа, содержать описание различных вариантов решения, обоснование выбранных решений.

\section{Разработка алгоритма решения}

\section{Описание программного обеспечения}

В следующем разделе можно привести описание программного обеспечения, особенностей его
реализации, связанных с выбранными инструментальными средствами и аппаратурой, на которой
оно призвано функционировать, накладываемых на него ограничений, установленных в исходных
данных к работе. Этот раздел не должен представлять собой перечень функций того или иного
программного модуля или инструкцию по использованию программного обеспечения.

\section{Сборка программного обеспечения}
	
	
\chapter{Тестирование}
	\section{JUnit}
\section{Тестирование реальной модели}
Раздел должен содержать описание выбора и обоснование
использовавшихся тестов, результаты автономной отладки отдельных модулей и отладки всего
комплекса программ, инструкции пользователю и т.п. При составлении этого раздела следует
руководствоваться методикой, изложенной в [6]. Результаты отладки программ по возможности
документированы представлением листингов или копий экранов.

\chapter*{Заключение}
	\addcontentsline{toc}{chapter}{Заключение}
	Заключение должно в краткой форме отразить результаты выполнения поставленной задачи:
	количественные и качественные оценки разработанных программных или аппаратных средств,
	достоинства и недостатки выбранных методов решения задачи. В нем приводятся сведения о
	практическом использовании выполненной работы (ссылки на акты о внедрении результатов,
	официальные протоколы испытаний, подготовленные или опубликованные статьи и научные отчеты,
	ссылки на конференции, семинары или выставки, в которых принимал участие автор). В приложении
	к пояснительной записке целесообразно приложить копии упоминаемых материалов. В заключении
	следует отметить перспективу развития работ, проводившихся во время дипломного проектирования.


\chapter*{Литература}
	\addcontentsline{toc}{chapter}{Литература}

\chapter*{Приложение}
	\addcontentsline{toc}{chapter}{Приложение}
	\section{Программа отладки на Java}
	\section{Программа управления на C}

\end{document}
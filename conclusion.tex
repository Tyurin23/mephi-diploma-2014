Заключение должно в краткой форме отразить результаты выполнения поставленной задачи:
	количественные и качественные оценки разработанных программных или аппаратных средств,
	достоинства и недостатки выбранных методов решения задачи. В нем приводятся сведения о
	практическом использовании выполненной работы (ссылки на акты о внедрении результатов,
	официальные протоколы испытаний, подготовленные или опубликованные статьи и научные отчеты,
	ссылки на конференции, семинары или выставки, в которых принимал участие автор). В приложении
	к пояснительной записке целесообразно приложить копии упоминаемых материалов. В заключении
	следует отметить перспективу развития работ, проводившихся во время дипломного проектирования.